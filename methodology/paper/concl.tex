\section{讨论和总结}
\label{sec:diss-and-concl}
本文基于分布式假设,对一些经典的词向量模型进行了分类,
分析和讨论。不同于单一的分类标准,本文从训练方法,
目标函数和模型所利用的分布式信息三个方面对模型进行分类,
从多个维度展现了模型的异同。对每一个模型,本文详细分析了
它的体系结构和计算复杂度,讨论了它的优点和局限性。
基于本文提出的分类法,本章将讨论几个词向量模型的可能的发展方向。

\subsection{把语段关系引入计数型模型}
计数型模型直接利用了词--词共现矩阵,因此它直接捕获了范式信息。
然而,\cref{subsec:syntag-paradig} 的分析表明,引入语段关系
能够增强模型的表达力。因此,语段关系可能对计数型模型有促进作用。

\subsection{结合预测型和计数型模型}
预测型和计数型模型本质上都用到了词--词共现数据\cite{pennington2014glove},
可见它们不是对立的。正如\cite{pmlr-v22-bordes12}结合了语段关系和范式关系,
我们可以结合预测型和计数型模型。例如,用计数型模型习得的词向量作为预测型
模型的初始化参数。

\subsection{发掘一词多义这一语言现象}
分别式假设告诉我们,具有相似分布规律的语言实体也具有相似的意思。
Sahlgren在\cite{Sahlgren2008}中进一步把分布式含义解释为语言实体的功能的区别。
于是我们可以这样说:一个语言实体具有含义,而这个含义正是它和其他实体
在分布上的差异。然而,分布式假设似乎没有涉及一个语言实体具有多个含义的情况。

另一方面,一词多义是一个常见的语言现象,一个词在不同上下文中可以表达多种含义,
这些含义不但具有语法的区别,还具有语义的区别。
例如:在句子An old man fell to the ground和句子How did you get along with the old man?
中,old man分别表示老人和父亲(北美俚语),是两个完全不同的含义。
人能够快速的识别一词多义,而对于向量空间中的词向量,
我们可以想象,一个向量的位置可能是它的不同含义导致的位置的平均值。

在一词多义的情况下,词的含义必须放到上下文中考察,这可能导致词向量的衡量方法的改革。
另一方面,目前的词向量模型把词和它的表示看成是一一对应关系;
假如我们允许一个词对应多个词向量——不妨称之为词矩阵,那么一词多义就有了一种自然的表达,
因为词矩阵的一个分量对应着词的一个含义。这可能导致词向量模型的改革。
