\begin{abstract}
  本文回顾和总结了计算引论的内容。计算引论是一门介绍计算理论的入门课程,包含了多个子学科:
  可计算性理论,计算复杂度理论,数理逻辑,上下文无关语言以及并行计算模型。
  本课程对每一个领域解决的问题或者提出的定理或模型进行了介绍,着重介绍了其中的算法
  或推理过程,有鉴于此,本文对课程知识的回顾也从 {\itshape 领域的问题,定理和模型、算法和
  推理过程} 这几个方面进行。
\end{abstract}

\chapter{引言}
计算引论开设的背景是计算机行业知识的更新换代非常快,相比之下,
计算机基础理论就显得格外重要。
本课的目的是使学生从专业角度掌握计算机专业中计算的基本概念和原理,
培养学生的计算机专业素养,特别是从专业理论基础上,培养学生的专业思维方式和解决问题的方法,拓展学生的思考问题空间,
为系统地掌握计算机相关领域的专业知识奠定理论基础。
本课的意义是,通过本课程的学习,学生能够判断问题的难度,
并且能够具备通过建立相应的计算模型,专业地解决问题的能力,养成专业思考的能力。
