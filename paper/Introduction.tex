\chapter{绪论}
广义说,计算是信息处理的科学名词。 狭义说,计算是指算术、代数和微积分等。
计算的科学定义是:建立模型、进行模拟、实验、验证等\cite{DBLP:books/daglib/0086373}。
上世纪初,德国大数学家希尔伯特(Hilbert)提出:
\begin{quotation}
        是否存在着一个通用过程,这个过程能用来判定任意数学命题是否成立,即,
        输入一个数学命题,在有限时间内,如果这个命题成立,得到一个证明;或如果这个命题不成立,则得到一个反例。
\end{quotation}
图灵证明了对于平面几何来说,存在这样的过程。但是,对于一般的数学命题,不存在这样的过程。

\section{计算与计算模型}

%% lambda演算的提出者church是图灵的导师,lambda演算是“图灵完备”的,并且在图灵机
%% 之前,所以列在图灵机前面。
\subsection{Lambda演算}
\def\LambdaCalculus{$\lambda$ 演算}
\def\LambdaTerm{lambda项}

\LambdaCalculus (lambda calculus)是一套从数学逻辑中发展,以变数绑定和替换规则来研究函数
如何抽象化定义、函数如何被应用以及递归的形式系统。它由数学家阿隆佐$\cdot$ 邱奇(Church, Alonzo)
在20世纪30年代首次 发表。\LambdaCalculus 作为一种应用广泛的计算模型,可以清晰的定义什么是一个可计算函数。
而任何可计算函数都能以这种形式表达和求值,它能模拟单带图灵机的计算过程;尽管如此,\LambdaCalculus
强调的是变换规则的运用,而非实现它们的具体机器。

\LambdaCalculus 可以比拟最根本订单编程语言,它包括了一条变换规则和一条将函数抽象化
定义的方式,因此普遍公认是一种更接近软件而非硬件的方式。它对函数式编程造成了很大的
影响,比如Lisp、ML语言和Haskell语言。在1936年邱奇利用\LambdaCalculus 给出了对于
判定性问题的否定:关于两个lambda运算式是否等价的命题,无法由一个\emph{通用的算法}
判断。

\LambdaCalculus 包括了构建\LambdaTerm,对\LambdaTerm 执行规约的操作。
\begin{table}[H]
        \centering
        \caption{构建\LambdaTerm 的规则}
        \begin{tabular}{|l|l|l|}
                \hline
                语法 & 名称 & 描述 \\
                \hline
                $a$ & 变量 & 表示参数或数学/逻辑值的字元或字串 \\
                \hline
                $(\lambda x.M)$ & 抽象化 & 函数定义($M$是一个\LambdaTerm)。
                变量$x$在表达式中已被绑定。\\
                \hline
                $(M N)$ & 应用 & 将函数应用与参数。$M$ 和 $N$ 是\LambdaTerm。 \\
                \hline
        \end{tabular}
\end{table}

\subsection{图灵机}
\def\TuringMathDef{(Q, \Sigma, \Gamma, \delta, q_0, q_{accept}, q_{reject})}

图灵机 \cite{doi:10.1112/plms/s2-42.1.230} (Turing machine),
又称确定型图灵机,是英国数学家艾伦·图灵于1936年提出的一种抽象计算模型,
其更抽象的意义为一种数学逻辑机,可以看作等价于任何有限逻辑数学过程的逻辑机器。

\paragraph{图灵机的正式定义}
一台\textbf{图灵机}是一个七元有序组$\TuringMathDef$ ,其中 $Q$, $\Sigma$, $\Gamma$ 都是\emph{有限集合}, 且满足:
\begin{enumerate}
        \item $Q$ 是非空有穷状态集合;
        \item $\Sigma$ 是非空有穷输入字母表,其中不包含特殊的空白符 $\epsilon$;
        \item $\Gamma$ 是非空有穷带字母表且$\Sigma \subset \Gamma$;
        \item $\epsilon \in \Gamma - \Sigma $为\emph{空白符},
                也是唯一允许出现无限次的字符;
        \item $ \delta: Q \times \Gamma \rightarrow Q \times \Gamma \times \{L,R,X\} $
                其中 $L, R$ 表示读写头是向左移还是向右移, $X$ 表示不移动;
        \item $q_0 \in Q$ 是起始状态;
        \item $q_{accept} \in Q$ 是接受状态。$q_{reject} \in Q$ 是拒绝状态,
                且$q_{reject} \neq q_{accept}$。
\end{enumerate}

图灵机$ M = \TuringMathDef $将以如下方式运作:
刚开始的时候将输入符号串$\omega = \omega_0\omega_1\cdots\omega_{n-1} \in \Sigma^{*}$
从左到右依次填在纸带的$0,1,\cdots,n-1$号格子上,其它格子保存空白 (即填空白符$\epsilon$)。
$M$的读写头指向第0号格子,$M$处于状态$q_0$。
机器开始运行后,按照转移函数$\delta$所描述的规则进行计算。例如,若当前机器状态为$q$,读写头所指
的格子中的符号为$x$,设$\delta(q,x) = (q',x',L)$,则机器进入新状态$q'$, 将读写头所
指的格子中的符号改为$x'$,然后将读写头向左移动一个格子。若在某一时刻,读写头所指的
是第0号格子,但根据转椅函数它下一步将继续向左移,这时它停在原地不动。换句话说,读写头
始终不移出纸带的边界。若某个时刻$M$根据转移函数进入了状态$q_{accept}$,则它立刻停机 并接受输入的字符串;
若在某个时刻$M$根据转移函数进入了状态$q_{reject}$,则它立刻停机并且拒绝输入的字符串。

图灵机的变体和应用以及它在计算机理论和实际中的意义将在\cref{sec:TuringMachine}中详细讨论。

\subsection{邱奇--图灵论题}
邱奇-图灵论题(Church--Turing thesis)是一个关于可计算性理论的假设。
该假设论述了关于函数特性的,可有效计算的函数值(用更现代的表述来说——在算法上可计算的)。简单来说,
邱奇-图灵论题认为``任何在算法上可计算的问题同样可由图灵机计算''。

20世纪上半叶,对可计算性进行公式化表示的尝试有:
\begin{itemize}
        \item 美国数学家阿隆佐·邱奇创建了称为\LambdaCalculus 的方法来定义函数。
        \item 英国数学家阿兰·图灵创建了可对输入进行运算的理论机器模型,现在被称为通用图灵机。
        \item 邱奇等人一起定义了一类函数,这种函数的值可使用递归方法计算。
\end{itemize}

这三个理论在直觉上似乎是等价的——它们都定义了同一类函数。因此,计算机科学家和数学家们相信,可计算性的精确定义已经出现。
邱奇--图灵论题的非正式表述说:如果某个算法是可行的,那这个算法同样可以被图灵机,以及另外两个理论所实现。

虽然这三个理论被證明是等价的,但是其中的前提假设——``能够有效计算''是一个模糊的定义。
因此,虽然这个假说已接近完全,但仍然不能由公式证明。

图灵在1936年的论文\cite{doi:10.1112/plms/s2-42.1.230}中试图通过引入图灵机来形式地展示这一想法。
在论文中,他证明了``判定性问题''是无法解决的。
\emph{几个月之前},阿隆佐·邱奇在\cite{DBLP:journals/jsyml/Church36}中证明出了一个相似的论题,
但是他采用递归函数和\LambdaCalculus 来形式地描述有效可计算性。
\LambdaCalculus 由阿隆佐·邱奇和斯蒂芬·克莱尼提出\cite{10.2307/2372027,bernays_1936},
而递归函数由库尔特·哥德尔(Kurt G\"odel)和雅克·埃尔布朗(Jacques Herbrand)提出。
这两个机制描述的是同一集合的函数,正如邱奇和克林所展示的正整数函数那样。
在听说了邱奇的建议后,图灵很快就证明了他的图灵机实际上描述的是同一集合的函数。

\subsection{\VonArch}
\def\VonNeumann{冯·诺依曼}
\def\VonArch{\VonNeumann 体系结构}

1946年,\VonNeumann 用电子器件物理实现了图灵的计算模型,建成了世界的第一台计算机。
\textbf{\VonArch}\cite{DBLP:journals/annals/Neumann93}。 (Von Neumann architecture),
又称\textbf{普林斯顿结构} (Princeton architecture),
是一种将程序指令储存器和数据存储器合并在一起的电脑设计概念结构,
依照\VonArch 设计出的计算机又称\textbf{储存程序式计算机}。

最早的计算机器仅内含固定用途的程序。现代的某些计算器依然维持了这样的设计,只能完成特定
的任务。若想改变此机器的程序,必须更改线路、更改结构甚至重新设计机器。而储存程序型
电脑的概念改变了这一切。藉由创造一组指令架构,并将所谓的运算转化为一串程序指令的执行
细节,让机器更加灵活。储存程序计算机在体系结构上主要特点有:
\begin{enumerate}
        \item    以运算单元为中心
        \item    采用存储程序原理
        \item    存储器是按地址访问、线性编址的空间
        \item    控制流由指令流产生
        \item    指令由操作码和地址码组成
        \item    数据以二进制编码
\end{enumerate}

\section{计算能力与计算效率}
一个计算模型的计算能力是用它可计算的问题类的大小来刻画的。
目前人类尚无找到其它的计算模型,其可计算的问题类超过图灵机的计算能力。
计算效率用算法的时间需求(或空间需求)来刻画。

\subsection{P与NP问题}
非定常多项式(non-deterministic polynomial,NP)时间复杂性类,或称非确定性多项式时间复杂性类,
包含了可以在多项式时间内,对一个判定性算法问题的实例,一个给定的解是否正确的算法问题。

NP是计算复杂性理论中最重要的复杂性类之一。它包含复杂性类$P$,
即在多项式时间内可以验证一个算法问题的实例是否有解的算法问题的集合;
同时,它也包含NP完全问题,即在NP中“最难”的问题。计算复杂性理论的中心问题,
$ P/NP $问题即是判断对任意的NP完全问题,是否有有效的算法,或者$NP$与$P$是否相等。

\paragraph{形式化定义}
目前常用的定义是所谓的“关系定义式”。即对于一个语言$L$,$L$在$NP$中,
那么存在多项式时间图灵机$M$,和多项式$p$使得\cite{DBLP:journals/jacm/Ladner75}:
\begin{equation*}
        x \in L \iff \exists y \in \{0,1\}_{p(|x|)}, M(x,y) = 1
\end{equation*}

\begin{figure}[H]
        \caption{NP,P,NPC问题韦恩图}
        \centering
        \includesvg[width=0.5\textwidth]{./figure/Complexity_classes.svg}
\end{figure}

\paragraph{NP完全问题}
NP完全问题(NP-Complete, NPC),是计算复杂度理论中,决定性问题的等价之一。NPC问题
是NP(非决定性多项式时间)中最难的问题。因此NP完全问题应该是最不可能被化简为P
(多项式时间可决定)的决定性问题的集合。若\emph{任何}NPC问题得到多项式时间的解法,
那此解法就可以应用在所有NP问题上。

一个决定性问题$C$若是NPC,则表示它对NP是完备的,则:
\begin{enumerate}
        \item 它是一个NP问题,且
        \item 其他属于$NP$的问题都可以\emph{规约}成它。
\end{enumerate}
\textbf{可归约} 在此表示对每个问题$L$,总有一个多项式时间多对一变换,即一个决定性的
算法可以将实例$ l \in L $转化为实例$ c \in C $,并让$c$回答Yes当且仅当此答案对$I$也是
Yes \cite{DBLP:journals/jacm/Ladner75}。
第一个NPC问题是由Leonid Levin和Cook独立证出SAT问题是NPC问題\cite{DBLP:conf/stoc/Cook71}。
此后,许多问题被陆续证明为是NP完全问题。
% 这里以后如果字数不够可以在加一些。
一些经典的NPC问题:
\begin{itemize}
        \item 布尔可满足性问题
        \item N-puzzle问题(华容道问题)
        \item 哈密尔顿圆圈问题
        \item 邮递员问题
        \item 子图同构问题
        \item 分团问题
        \item 顶点覆盖问题
        \item 独立顶点集问题
        \item 图着色问题
\end{itemize}

\begin{figure}[H]
        \caption{NPC问题的推导关系}
        \centering
        \includesvg[width=0.5\textwidth]{./figure/Relative_NPC_chart.svg}
\end{figure}

\subsection{算法分析}
在计算机科学中,算法分析(Analysis of algorithm)
是分析执行一个给定算法需要消耗的计算资源数量(例如计算时间,存储器使用等)的过程\cite{DBLP:books/daglib/0023376}。
算法的效率或复杂度在理论上表示为一个函数。其定义域是输入数据的长度(通常考虑任意大的输入,没有上界),
值域通常是执行步骤数量(时间复杂度)或者存储器位置数量(空间复杂度)。算法分析是计算复杂度理论的重要组成部分。
理论分析常常利用渐近分析估计一个算法的复杂度,并使用大$O$符号、大$\Omega$符号和大$\Theta$符号作为标记。

举例,二分查找所需的执行步骤数量与查找列表的长度之对数成正比,记为 $ O(\log n) $
简称为``对数时间''。通常使用渐近分析的原因是,同一算法的不同具体实现的效率可能有差别。
但是,对于任何给定的算法,所有符合其设计者意图的实现,它们之间的性能差异应当仅仅是一个系数。

精确分析算法的效率有时也是可行的,但这样的分析通常需要一些与具体实现相关的假设,称为计算模型。
计算模型可以用抽象机器来定义,比如图灵机。或者可以假设某些基本操作在单位时间内可完成。

假设二分查找的目标列表总共有$n$个元素。如果我们假设单次查找可以在一个时间单位内完成,
那么至多只需要 $ \log n + 1 $ 单位的时间就可以得到结果。这样的分析在有些场合非常重要。

算法分析在实际工作中是非常重要的,因为使用低效率的算法会显著降低系统性能。
在对运行时间要求极高的场合,耗时太长的算法得到的结果可能是过期或者无用的。低效率算法也会大量消耗计算资源。

\section{计算理论的地位}
计算理论奠定了一切可计算问题的基础,它对计算机可以实际求解的问题的界限进行了划定。它为理解算法的
时间和空间复杂度,提高算法的性能提供了理论支持。它也为上下文无关语言提供了理论基础,形成了各种
编程语言语法的基础。它还在早期的人工智能发挥了一定的作用。
